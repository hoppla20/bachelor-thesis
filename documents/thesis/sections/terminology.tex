\chapter{Terminology}

\section{Terms}

\subsection{Platform}

Group of technologies and services that are used as a base upon which
applications are developed and deployed.

\todo[inline]{Expand explanation of term `Platform'}
\todo[inline]{Go more into dynamic infrastructure management}

\subsection{Confidentiality}
\label{sec:confidentiality}

Isolation techniques like containerization and virtualization protect the
infrastructure or platform from applications. On the other hand confidentiality
is the direct opposite and its goal is to protect the application from the
infrastructure or platform it is running on.

\subsection{Trusted Computing Base (TCB)}

The set of all hardware, firmware, and/or software components that are critical
to the security of a given computing system. These components are the only
components in the computing system that operate at a high level of trust. This
does not imply that these components are secure, but that they are crucial to the
security of system as a whole.

\section{Roles}

While this section defines roles and their tasks, one should keep in mind that a
single entity can take on multiple roles. Section \ref{sec:trust-model} will
define which roles shouldn't be aggregated onto one entity.

\todo[inline]{Remove term ``client'' from document}

\subsection*{Infrastructure Provider}

The infrastructure provider controls the hardware and firmware used to provide
compute, networking, and storage resources.

\subsection*{Service Provider}

An entity providing services on top of infrastructure provided by the
infrastructure provider in order to ease the development and deployment of
applications -- often in the form of a platform.

\subsection*{Application Owner}

An entity developing an application for a customer on the platform provided by
the service provider.

\subsection*{Data Owner}

An entity that is in the possession of possibly sensible data that will be
processed or used by an application deployed on the platform provided by the
service provider.

\subsection*{Verifier}

An entity that verifies that the services provided by the service provider can
be trusted.
