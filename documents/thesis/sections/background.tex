\section{Cryptographic Concepts}
\label{sec:cryptographic-concepts}

Cryptography is the practice of protecting data by transforming it into a format
that is unreadable by unauthorized parties. In this background section, we will
look at the basics of symmetric cryptography, asymmetric cryptography, and key
agreement protocols.

\subsection{Symmetric and Asymmetric Cryptography}
\label{sec:symmetric-asymmetric-cryptography}

Symmetric cryptography, also known as secret-key cryptography, is a method of
encryption where the same key -- the secret key -- is used for encryption and
decryption. The key has to be securely distributed and stored to prevent
unauthorized access. The most common symmetric encryption algorithms are
Advanced Encryption Standard (AES) and Data Encryption Standard (DES).

The main benefit of symmetric cryptography is its performance. Since the same
key is used for encryption and decryption, it is fast and efficient. However,
one of the significant drawbacks of symmetric cryptography is the issue of key
management. Securely storing and distributing keys to entities and components
that require access to encrypted data is challenging. Furthermore, attackers
that gain access to a key can quickly gain plain-text access to encrypted
confidential data.

Asymmetric cryptography, also known as public-key cryptography, is a method of
encryption where two different keys are used for encryption and decryption. One
key is kept private, and the other key is made public. The private key is used
for decryption, while the public key is used for encryption. These keys are
mathematically related, but the private key cannot be derived from the public
key. The most common asymmetric encryption algorithms are RSA and Elliptic Curve
Cryptography (ECC).

Digital signatures are a widely used application of asymmetric cryptography. A
digital signature is a cryptographic primitive that verifies a dataset's
integrity and authenticity. The data owner uses their private key to generate a
unique digital signature cryptographically bound to the dataset's content. When
the owner sends the dataset together with the signature to another party, the
receiver can use the owner's public key to verify the signature's authenticity
and the dataset's integrity. A successfully validated signature gives the
recipient confidence that the data owner sent the dataset and that the dataset
has not been modified during transit.

Asymmetric cryptography is slower than symmetric cryptography due to the
complexity of the mathematical algorithms involved. One of the significant
benefits of asymmetric cryptography is that it eliminates the need for secret
key management. Each user or system can have a unique pair of public and private
keys, and the public key can be shared with anyone. The private key, on the
other hand, has to be stored securely.

\subsection{Key Agreement}
\label{sec:key-agreement}

Key agreement is a crucial component of secure communication. It involves two
parties agreeing on a secret key (symmetric cryptography) that can be used for
the encryption and decryption of messages. This section illustrates a key
agreement process using the popular Diffie-Hellman protocol.

The Diffie-Hellman protocol is a key agreement protocol that utilizes asymmetric
cryptography in order to allow two parties to agree on a mutual secret key over
an insecure channel. A simplified overview of the process is as follows (with
parties Alice and Bob):

\begin{enumerate}
  \item Both parties establish shared parameters: prime number $p$ and generator
        $g$.
  \item Each party generates a random number ($A$ for Alice and $B$ for Bob) and
        computes a publicly known number $pub_A = g^A \mod p$ and $pub_B = g^B
          \mod p$ respectively.
  \item Both parties exchange their public number $pub_A$ and $pub_B$.
  \item Both parties compute the shared secret key $K = (pub_A)^B \mod p =
          (pub_B)^A \mod p = g^{AB} \mod p$
  \item Alice and Bob can then use the secret key $K$ to encrypt and decrypt
        messages sent between both parties, establishing a secure communication
        channel over an insecure channel.
\end{enumerate}

Even though attackers might gain knowledge about $p$, $g$, $pub_A$, and $pub_B$
by listening to the insecure channel, they can not derive the secret key $K$
from the publicly known numbers. Suppose we assume that the prime number p and
generator g are already specified in a specific protocol or have been
pre-established. In that case, Alice and Bob have to exchange exactly two
messages, $pub_A$ and $pub_B$.

However, an attacker might perform a man-in-the-middle attack, where the
attacker intercepts the key agreement process and establishes shared secrets
with both Alice and Bob. If Alice sends Bob an encrypted message, the attacker
decrypts the message using the shared secret key with Alice, gains access to the
plain text content of the message, encrypts the message using the shared secret
with Bob, and finally relays the message to Bob.

Not only does the attacker gain plain-text access to the messages, but the
attacker can even impersonate either party in the communication. For example,
Bob thinks that the secret key he/she established with the attacker is only
known by Bob and Alice. Therefore, Bob assumes that messages encrypted with this
secret key are sent by Alice, although the attacker might have sent them.

To mitigate man-in-the-middle attacks, either Alice or Bob must authenticate the
other party. This can be achieved by utilizing digital signatures. For example,
Bob uses his own private key to sign the message that includes $pub_B$ and sends
the message and the signature to Alice. Alice then uses Bob's public key to
verify the messages authenticity and integrity.

Public keys can be securely exchanged between Alice and Bob by facilitating a
Public Key Infrastructure (PKI). However, this thesis will not further go into
detail about PKIs.
