\section{Cryptographic Concepts}
\label{sec:cryptographic-concepts}

Cryptography is the practice of protecting data by transforming it into a format
that is unreadable by an unauthorized party. In this background section we will
have a look at the basics of symmetric cryptography, asymmetric cryptography,
and key agreement protocols.

\subsection{Symmetric and Asymmetric Cryptography}
\label{sec:symmetric-asymmetric-cryptography}

Symmetric cryptography, also known as secret-key cryptography, is a method of
encryption where the same key is used for both encryption and decryption. This
means that both the sender and receiver must have the same key to encrypt and
decrypt messages. In symmetric cryptography, the key is kept secret and secure
to prevent unauthorized access. The most common symmetric encryption algorithms
are Advanced Encryption Standard (AES) and Data Encryption Standard (DES).

The main benefit of symmetric cryptography is its performance. Since the same
key is used for encryption and decryption, it is a fast and efficient process.
However, one of the major drawbacks of symmetric cryptography is the issue of
key management. Managing and distributing keys to entities and components that
require access to encrypted data is a challenging task. Furthermore, if a key is
compromised, all data encrypted with that key can be easily decrypted.

Asymmetric cryptography, also known as public-key cryptography, is a method of
encryption where two different keys are used for encryption and decryption.
These keys are mathematically related, but they cannot be derived from each
other. One key is kept private, and the other key is made public. The private
key is used for decryption, while the public key is used for encryption. The
most common asymmetric encryption algorithms are RSA and Elliptic Curve
Cryptography (ECC).

One of the significant benefits of asymmetric cryptography is that it eliminates
the need for key management. Each user or system can have a unique pair of
public and private keys, and the public key can be shared with anyone. The
private key, on the other hand, is kept secure and confidential. Asymmetric
cryptography is slower than symmetric cryptography due to the complexity of the
mathematical algorithms involved. However, it is more secure since even if the
public key is compromised, the private key cannot be derived from it.

Digital signature is a widely used application of asymmetric cryptography. A
digital signature is a cryptographic primitive that is used to verify the
integrity and authenticity of a data. The process of digital signature involves
two steps: signing and verification. To sign data, the owner of the data uses
their private key to generate a unique digital signature. When the owner sends
the data together with the signature to another party, the receiver can use the
owner's public key to verify the signature's authenticity and the data's
integrity. If the verification of the digital was successful, the receiver can
be confident that the data was sent by the owner has not been modified during
transit.

\subsection{Key Agreement}
\label{sec:key-agreement}

Key agreement is a crucial component of secure communication. It involves two
parties agreeing on a secret key (symmetric cryptography) that can be used for
encryption and decryption of messages. One of the most popular key agreement
protocols is the Diffie-Hellman protocol.

The Diffie-Hellman protocol is a key agreement protocol that utilizes asymmetric
cryptography in order to allow two parties to agree on a mutual secret key over
an insecure channel. A simplified overview of the process is as follows (with
parties Alice and Bob):

\begin{enumerate}
  \item Both parties establish shared parameters: a prime number $p$ and
        generator $g$.
  \item Each party generates a random number ($A$ for Alice and $B$ for Bob) and
        computes a publicly known number $pub_A = g^A \mod p$ and $pub_B = g^B
          \mod p$ respectively.
  \item Both parties exchange their public number $pub_A$ and $pub_B$.
  \item Both parties compute the shared secret key $K = (pub_A)^B = (pub_B)^A =
          g^{AB} \mod p$
  \item The secret key $K$ can then be used to encrypt and decrypt messages send
        between both parties, establishing a secure communication channel over
        an insecure channel.
\end{enumerate}

Even though attackers might gain knowledge about $p$, $g$, $pub_A$, and $pub_B$
by listening to the insecure channel, they can not derive the secret key $K$
from the publicly known numbers. If we assume that the prime number $p$ and
generator $g$ are already specified in a specific protocol or have been
pre-established, there are exactly two messages that need to be exchanged
between Alice and Bob. For example, if Alice is the initiator of the
communication, Alice starts by sending $pub_A$ upon which Bob responds with
$pub_B$. After exchanging those two messages, Alice and Bob can already derive
the shared secret key $K$.

However, an attacker might perform a man-in-the-middle attack, where the
attacker intercepts the key agreement process and establishes shared secrets
with both Alice and Bob. If Alice sends Bob an encrypted message, the attacker
decrypts the message using the shared secret key with Alice, gains access to the
plain text content of the message, encrypts the message using the shared secret
with Bob, and finally relays the message to Bob.

Furthermore, the attacker can even impersonate either party in the
communication. For example, Bob thinks that the secret key he/she established
with the attacker is only known by Bob and Alice. Therefore, Bob assumes that
messages encrypted with this secret key and send to Bob by the attacker, are
sent by Alice.

In order to mitigate man-in-the-middle attacks, either Alice or Bob has to
authenticate the other party. This can be achieved using asymmetric
cryptographic keys. For example, Bob uses his own private key to sign the
message that includes $pub_B$ and is sent to Alice. Alice then uses Bob's public
key in order to verify that the message is indeed sent by Bob. The public keys
can be securely exchanged between Alice and Bob by facilitating a Public Key
Infrastructure (PKI). However, this thesis will not further go into detail about
PKIs.
