%% LaTeX2e class for student theses
%% sections/abstract_en.tex
%% 
%% Karlsruhe Institute of Technology
%% Institute for Program Structures and Data Organization
%% Chair for Software Design and Quality (SDQ)
%%
%% Dr.-Ing. Erik Burger
%% burger@kit.edu
%%
%% Version 1.3.6, 2022-09-28

\Abstract

In the current computing landscape distributed computing systems, largely based
on Grid and Cloud computing, have become the main ways of sharing infrastructure
resources, such as compute, storage, and network resources, while also providing
services that ease the development and orchestration of applications on said
resources. On the one hand, the usage of these resources and service come with
benefits such as higher availability, reducing complexity of applications, and
cost-efficiency, on the other hand, traditionally these benefits come with the
cost of fully trusting the provider of the resources and services with
potentially confidential data. Nevertheless, as consumer and government demands
for data privacy increase (e.g., GDPR coming into effect in the EU in 2018), the
distributed computing model must adapt to meet the increasingly strict trust
requirements. The advent of confidential computing enables a new distributed
computing model where the provider of infrastructure resources becomes untrusted
by providing hardware-based trusted execution environments (TEEs). This thesis
researches the current state of trusted execution environment technologies,
remote attestation procedures that allow tenants to verify the trustworthiness
of TEEs, and introduces a new \textit{trusted distributed computing model} that
integrates these two concepts into the traditional distributed computing model
in order to remove the provider of the distributed computing system from the
list of trusted parties.
