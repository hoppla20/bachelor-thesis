\Abstract

In the current computing landscape, distributed computing systems, primarily
based on Grid and Cloud computing, have become the main ways of sharing
infrastructure resources, such as compute, storage, and network resources while
also providing services that ease the development and orchestration of
applications on said resources. On the one hand, the usage of these resources
and services comes with benefits such as higher availability, reducing the
complexity of applications, and cost-efficiency, on the other hand, these
benefits come with the cost of fully trusting the provider of the resources and
services with potentially confidential data. However, as the demand from
consumers and governments for data privacy increases (e.g., GDPR coming into
effect in the EU in 2018), the distributed computing model must adapt to meet
the increasingly strict trust requirements. The advent of confidential computing
enables a new distributed computing model where the providers of resources and
services become untrusted by utilizing hardware-based trusted execution
environments (TEEs). This thesis researches the current state of confidential
computing technologies, remote attestation procedures that allow tenants to
assess the trustworthiness of TEEs, and introduces a new \textit{trusted
distributed computing model} that integrates these two concepts into the
traditional distributed computing to remove the provider of the distributed
computing system from the list of trusted parties.
