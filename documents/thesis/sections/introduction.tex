Today distributed computing systems are mainly based on a layered architecture
where each layer's security depends on the security of the layers beneath it.
The lowest layer is the hardware appliances that form the foundation of
distributed computing systems, and the highest layer is the collection of user
applications that implement abstract business logic. Each layer abstracts the
complexity of the layers below it and provides a simplified interface to the
layers above.

Service providers -- entities that provide the resources and services of a
distributed computing system -- are primarily concerned with the protection of
the components that make up the lower layers of distributed computing systems
(e.g., hypervisors providing virtualization of compute resources) from untrusted
code that tenants want to execute. Traditionally, this protection only isolates
the lower layer from the upper layer and does not protect the confidentiality of
the upper layer from the lower layer. As a result, the traditional distributed
computing model is completely built around the assumption that tenants trust the
service provider's software stack, including privileged software such as
hypervisors and firmware, but also the service provider's staff, including
system administrators and employees with physical access to hardware. But this
trust is not only present in compute, storage, and network resources, today many
applications utilize services provided by the service provider that implement
security functionality, such as authentication, authorization, and management of
cryptographic key material in order to decrease the complexity of the
application and be more cost-efficient.

Currently, the best practice for tenants to protect confidential data is to
manage cryptographic keys themselves, for example, by operating their own key
management service (KMS) or service provider offered hardware security module
(HSM). While this solution protects cryptographic keys from the service
provider, the keys are still used to decrypt confidential data that is
subsequently used by compute resources managed by the service provider, making
the protection of cryptographic keys from the service provider pointless. Other
solutions rely on new cryptographic primitives that allow specific compute
operations to be directly executed on encrypted data
\cite{monique2013homomorphicencryption, bellare2007searchableencryption}.
However, these solutions currently are either not applicable to general-purpose
computing or have very high performance limitations that make them impractical.

This thesis introduces the integration of confidential computing technologies
into the traditional distributed computing model. Confidential computing is an
arising technology that provides execution environments, so-called trusted
execution environments (TEEs), that protect applications by utilizing hardware
primitives. By implementing the protection of TEEs in the lowest layer, the
hardware, it is possible to remove management software such as hypervisors and
operating systems from the list of required trusted software components and
service providers from the list of required trusted entities. While TEEs protect
applications and confidential data, tenants have to be able to verify that
applications are running inside TEEs and that the service provider has not
modified the TEE platform. Remote attestation is an already widely used method
for assessing the trustworthiness of remote devices, components, and
environments. Commercially available confidential computing technologies such as
Intel SGX and AMD SEV have built-in support for producing evidence indented for
remote attestation purposes.

This thesis will first research the current state of the traditional distributed
computing model, confidential computing technology, and remote attestation
procedures in Chapter \ref{ch:technical-research} and subsequently discuss the
new \textit{trusted distributed computing model} in which service providers are
considered untrusted in Chapter \ref{ch:trusted-distributed-computing-model}.
Section \ref{sec:traditional-distributed-computing} decomposes the two major
distributed computing models Cloud and Grid computing into a generalized
architecture, which is evaluated by Section \ref{sec:untrusted-threat-model} in
order to identify threats from untrusted service providers defining a threat
model. Chapter \ref{ch:trusted-distributed-computing-model} is mainly concerned
with the integration of confidential computing (Section
\ref{sec:confidential-computing}) and remote attestation (Section
\ref{sec:remote-attestation}) into the traditional distributed computing model
and defines requirements for the resulting trusted distributed computing model
in Section \ref{sec:requirements}. Lastly, this thesis will evaluate the
untrusted distributed computing model upon the threat model (Section
\ref{sec:evaluation}), and look at two case studies (Section
\ref{sec:case-studies}).
