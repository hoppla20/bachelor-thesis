%% LaTeX2e class for student theses
%% sections/content.tex
%% 
%% Karlsruhe Institute of Technology
%% Institute for Program Structures and Data Organization
%% Chair for Software Design and Quality (SDQ)
%%
%% Dr.-Ing. Erik Burger
%% burger@kit.edu
%%
%% Version 1.3.6, 2022-09-28

\chapter{Introduction}
\label{ch:Introduction}

\section{Motivation}
\label{sec:motivation}

\section{Goal}
\label{sec:goal}

This thesis focuses on two main goals:

\begin{enumerate}[label=\textbf{\Roman*}]
  \item Management of infrastructure and orchestration of application workloads
  running on this infrastructure. Provide application developers services that
  support them with the development of applications and the deployment of
  application workloads.
  \item Make application workloads deployed by these services confidential
  (definition in section \ref{sec:confidentiality}) and allow verification of
  this confidentiality by clients and third-parties.
\end{enumerate}

Services mentioned in goal \goalRef{1} include
\begin{itemize}
  \item continuous integration (CI)
  \item continuous delivery and deployment (CD)
  \item deploying applications like services and computation tasks
  \item orchestration and monitoring of those applications
\end{itemize}
While these services should support the application owner in developing and
deploying applications (\subGoalRef{1}{1}) it should impose as little
limitations as possible on the application owner. Decisions regarding the
programming language, libraries, and frameworks should be made by the
application owner (\subGoalRef{1}{2}).

Goal \goalRef{2} doesn't imply application security. Application workloads
deployed with the help of the service provider should be shielded from the
infrastructure, the service provider, and its services. But securing the
application is outside the scope of this goal (\subGoalRef{2}{1}). The
confidentiality should also not require manual application modification by the
application owner (\subGoalRef{2}{2}).

\section{Reference Use Cases}

\section{Research Methodology}
\label{sec:research-methodology}
