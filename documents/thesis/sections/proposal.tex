%% LaTeX2e class for student theses
%% sections/proposal.tex
%% 
%% Karlsruhe Institute of Technology
%% Institute for Program Structures and Data Organization
%% Chair for Software Design and Quality (SDQ)
%%
%% Dr.-Ing. Erik Burger
%% burger@kit.edu
%%
%% Version 1.3.6, 2022-09-28

\chapter{Secure Computation Platform}
\label{ch:proposal}

\section{Architecture}

\subsection{Design Choices}

\begin{table}
  \centering
  \small
  \setitemize{noitemsep,topsep=0pt,leftmargin=*}
  \begin{tabular}{L{75pt}|L{100pt}L{100pt}L{100pt}}
                                               & \multicolumn{1}{C{100pt}}{Infrastructure as a Service} & \multicolumn{2}{c}{Platform as a Service}                                                  \\
                                               &                                                        & \multicolumn{1}{c}{Container}                             & \multicolumn{1}{c}{Binary}     \\
    \hline
    \hline
    Provides                                   & VMs with CC enabled                                    & Container runtime that runs containers in a CC enabled VM & CC enabled runtime environment \\
    \hline
    Infrastructure \& Orchestration managed by & Application Owner                                      & CSP                                                       & CSP                            \\
    \hline
    Development Environment managed by         & Application Owner                                      & Application Owner                                         & CSP                            \\
    \hline
    Notes                                      &
    \begin{itemize}
      \item Everything has to be managed by
            the application owner
      \item Goal is to reduce complexity
    \end{itemize}      &
    \begin{itemize}
      \item Big attack surface (TCB) if done
            bad
      \item Much more in control of CSP
      \item Services supporting CC usage can be
            provided by CSP
    \end{itemize}  &
    \begin{itemize}
      \item Least control by application owner
      \item Either complex dependency management
            and language support by CSP
      \item Or language or dependency limitations
    \end{itemize}                                                                                                                                                       \\
  \end{tabular}
  \caption{An overview over different services models.}
  \label{table:service-models}
\end{table}
\todo{More description for table}

\subsubsection*{Containers}

\begin{itemize}
  \item Puts control over development environment into the hands of application
        owner
  \item No complex language support and dependency management by CSP
  \item No language/dependency limitations
\end{itemize}

\subsubsection*{PaaS Service Model}

Lets CSP support application developers
\begin{itemize}
  \item provide infrastructure and services that help in deployment
        \begin{itemize}
          \item image building
          \item image deployment
          \item image registry service
        \end{itemize}
  \item provide infrastructure and service that help in enabling and
        verifying CC
        \begin{itemize}
          \item gathering and sending hardware CC attestation results
          \item
        \end{itemize}
\end{itemize}

\subsubsection*{Kubernetes}

Kubernetes is ``a portable, extensible, open source platform for managing
containerized workloads and services''. It provides some generally applicable
PaaS features such as deployment, scaling, load balancing, while being very
extensible to enable integration with for example logging, monitoring, alerting,
etc. Kubernetes is designed to enable user choice and flexibility, so instead of
it being a monolithic platform every solution is optional and pluggable. This
enables a wide variety of supported infrastructure platforms. \todo{Why not
  docker swarm?}

\section{Implementation Challenges}
