%% LaTeX2e class for student theses
%% sections/proposal.tex
%% 
%% Karlsruhe Institute of Technology
%% Institute for Program Structures and Data Organization
%% Chair for Software Design and Quality (SDQ)
%%
%% Dr.-Ing. Erik Burger
%% burger@kit.edu
%%
%% Version 1.3.6, 2022-09-28

\chapter{Secure Computation Platform}
\label{ch:proposal}

\section{Goal}

Provide a platform for
\begin{itemize}
  \item continuous delivery and deployment (CD)
  \item deploying services
  \item running computation tasks and gathering results
  \item gathering logs and metrics
\end{itemize}
in a secure environment:
\begin{itemize}
  \item Data protection (confidentiality and integrity) in every state
  \item Code integrity
  \item Automated remote attestation
\end{itemize}

The goal is not to promise security in every software layer but a trust model
which separates the infrastructure layers from the guest application.

The CSP should manage the infrastructure and orchestrate the applications so
that the application owner can focus on developing the application instead of
worrying about the infrastructure. The orchestration of applications should only
include unprivileged actions.

But the security of the application is not part of this platform and should be
handled by the application owner. While supporting the developers in deploying
their applications the platform should impose as little limitations as possible.
The application developer should be able to make own decisions regarding the
programming language and the libraries they want to use.

Furthermore, data owners should not blindly trust the CSP. Which means
application and data owners should be able to define security requirements and
verify them before sending/unlocking sensitive data. These requirements include
\begin{itemize}
  \item Software running in confidential VMs including initrd and kernel
  \item Hardware Confidential Computing support
  \item Firmware (OVMF)
\end{itemize}

\section{Architecture}

\subsection{Design Choices}

\begin{table}
  \centering
  \small
  \setitemize{noitemsep,topsep=0pt,leftmargin=*}
  \begin{tabular}{L{75pt}|L{100pt}L{100pt}L{100pt}}
                                               & \multicolumn{1}{C{100pt}}{Infrastructure as a Service} & \multicolumn{2}{c}{Platform as a Service}                                                  \\
                                               &                                                        & \multicolumn{1}{c}{Container}                             & \multicolumn{1}{c}{Binary}     \\
    \hline
    \hline
    Provides                                   & VMs with CC enabled                                    & Container runtime that runs containers in a CC enabled VM & CC enabled runtime environment \\
    \hline
    Infrastructure \& Orchestration managed by & Application Owner                                      & CSP                                                       & CSP                            \\
    \hline
    Development Environment managed by         & Application Owner                                      & Application Owner                                         & CSP                            \\
    \hline
    Notes                                      &
    \begin{itemize}
      \item Everything has to be managed by
            the application owner
      \item Goal is to reduce complexity
    \end{itemize}      &
    \begin{itemize}
      \item Big attack surface (TCB) if done
            bad
      \item Much more in control of CSP
      \item Services supporting CC usage can be
            provided by CSP
    \end{itemize}  &
    \begin{itemize}
      \item Least control by application owner
      \item Either complex dependency management
            and language support by CSP
      \item Or language or dependency limitations
    \end{itemize}                                                                                                                                                       \\
  \end{tabular}
  \caption{An overview over different services models.}
  \label{table:1}
\end{table}
\todo{More description for table}

\subsubsection*{Containers}

\begin{itemize}
  \item Puts control over development environment into the hands of application
        owner
  \item No complex language support and dependency management by CSP
  \item No language/dependency limitations
\end{itemize}

\subsubsection*{PaaS Service Model}

Lets CSP support application developers
\begin{itemize}
  \item provide infrastructure and services that help in deployment
        \begin{itemize}
          \item image building
          \item image deployment
          \item image registry service
        \end{itemize}
  \item provide infrastructure and service that help in enabling and
        verifying CC
        \begin{itemize}
          \item gathering and sending hardware CC attestation results
          \item
        \end{itemize}
\end{itemize}

\subsubsection*{Kubernetes}

Kubernetes is ``a portable, extensible, open source platform for managing
containerized workloads and services''. It provides some generally applicable
PaaS features such as deployment, scaling, load balancing, while being very
extensible to enable integration with for example logging, monitoring, alerting,
etc. Kubernetes is designed to enable user choice and flexibility, so instead of
it being a monolithic platform every solution is optional and pluggable. This
enables a wide variety of supported infrastructure platforms. \todo{Why not
  docker swarm?}

\section{Implementation Challenges}
