\section{Existing Solutions}

\subsection{Commercial Solutions by well-known Service Providers}

\begin{description}
  \item[GCP Confidential Space]
    Let multiple parties share confidential data with a workload while retaining
    the confidentiality and ownership of the data.
  \item[GCP Confidential Dataproc]
    Big data processing through fully managed data processing frameworks and
    tools like Spark and Hadoop. Uses confidential VMs to provide
    confidentiality.
  \item[Azure Confidential Containers]
    \todo[inline]{Description: Azure Confidential Containers}
  \item[AWS Nitro]
    \todo[inline]{Description: AWS Nitro}
\end{description}

\todo[inline]{Problems with commercial solutions}

\subsection{Kubernetes as platform base}

Kubernetes is an extensible, open source platform for managing containerized
workloads and services. Due to its open source nature and popularity there has
been a rapidly growing ecosystem surrounding Kubernetes. It provides general
platform features such as deployment, orchestration, scaling, load-balancing,
integration with logging, monitoring, and alerting solutions. Because of its
pluggable and extensible design it has been widely adopted for building full
platform solutions.

Explanation of needed Kubernetes components:

\begin{description}
  \item[Node]
    Machine running workloads (application and control-plane workloads)
  \item[Pods]
    A group of containers that share storage and network resources that models
    an application-specific logical host. The containers that make up a pod are
    always located on the same node and are scheduled in unison.
\end{description}

\todo[inline]{Expand as needed in the next sections}
\todo[inline]{Explain etcd, control-plane, worker nodes (data, control, workload plane)}

In the Kubernetes architecture there are three levels on which confidentiality
can be applied to: The node, pod, or container. There are distinct benefits and
problems for applying confidentiality on each layer. We will discuss them in the
following sections.

\subsubsection{Confidential Nodes}

The most outer layer where confidentiality can be applied to in the Kubernetes
architecture are the nodes. By using confidential computing enabled virtual
machines -- for example by facilitating AMD SEV or Intel TDX -- a Kubernetes
cluster operator is able to shield workloads running inside the cluster.
Prominent service providers like Azure and GCP already offer the option of
deploying their managed Kubernetes clusters with confidential worker nodes.
However, these solutions don't include verification of the nodes which means
that one would have to build a custom remote attestation system -- for example
by implementing the RATS (section \ref{sec:rats}).

A more complete solution would be Edgeless Systems' Constellation Kubernetes.
\todo[inline]{Go into the verification architecture of Constellation Kubernetes}

The biggest problem with confidential nodes is the restriction of cluster admin
privileges and verifying these as a client. While the workloads are shielded
from the infrastructure cluster administrators would still have full control
over the workloads running inside the cluster which breaks goal \goalRef{2}.

\subsubsection{Confidential Containers}

The other extreme would be to apply confidentiality to containers. A process
running inside a TEE would only receive confidential data like decryption keys
or personalized information after verifying that the process is running inside a
TEE via remote attestation. Since the TEE shields the process from the cluster
this approach remove the cluster administrator from the TCB of the application.

\todo[inline]{Example: Kata Containers with VM-based TEE}

Even though this approach doesn't conflict with the goals of this paper, it does
collide with the design of Kubernetes. In the Kubernetes architecture the
smallest deployable unit of computation is a pod, a collection of application
containers. These containers share storage and network resources, but it becomes
very hard to share these resources when the containers are shielded from each
other. The next approach addresses this architectural issue.

\subsubsection{Confidential Pods}
\label{sec:confidential-applications}

Instead of applying confidentiality to containers we can also shield pods from
the outside world. This improves the last approach as per definition a pod in a
Kubernetes cluster defines an application-specific logical host. As opposed to
shielding a single container, shielding a pod would allow sharing storage and
networking resources between containers composing the pod.

There is an ongoing effort to realize this approach at the project Confidential
Containers.

\todo[inline]{Go into details about Confidential Containers architecture.}
