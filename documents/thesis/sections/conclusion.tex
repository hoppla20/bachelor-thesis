The aim of this thesis was the introduction of a new trusted distributed
computing model that protects the confidentiality of tenant data from untrusted
service providers. Using a threat model that identified threats and issues that
arise when service providers become untrusted, this thesis evaluated the trusted
distributed computing model, which integrates confidential computing and remote
attestation into the traditional distributed computing model. The evaluation
showed that confidential computing combined with remote attestation could solve
the threats and issues of the traditional model.

However, the removal of the service provider as a trusted entity comes with a
trade-off. While the traditional model allowed tenants to shift more and more
responsibilities to the service provider, reducing the complexity of
applications, the trusted model reduces this benefit by giving responsibilities
that could compromise the confidentiality of data back to the application owner.

Additionally, current confidential computing technologies still have performance
limitations that either lessen the cost-efficiency aspect of facilitating
distributed computing systems or make using confidential computing technology
impractical.

Furthermore, confidential computing technologies are still relatively new.
Implementations of TEE platforms, remote attestation protocols, and their
support in other software components are still in the early stages of
development. Further research might uncover new vulnerabilities that require
changes in implementations and design choices. Therefore, protocols and
implementations are all still subject to substantial changes. While Cloud
providers are eager to adopt this new technology and are already offering
services based on confidential computing technologies, the two case studies have
shown that the infrastructure and platform layers are still not fulfilling all
requirements of the trusted distributed computing model.

Nevertheless, confidential computing provides very promising hardware-based
primitives that could allow a new generation of trustworthy distributed
computing systems that meet the increasing demand for more privacy.
